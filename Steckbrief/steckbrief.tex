\documentclass[ngerman]{article}
\title{Wegfindung im Labyrinth (AP)}
\author{
    Florian Nowak \& Yichuan Shen\footnote{Studienfächer: B.\,Sc. Mathematik (Florian) \&  M.\,Sc. Mathematik (Yichuan)}\\
    Betreuer: Gero Plettenberg, Thomas Kloepfer
}
\date{Wintersemester 2014/15}

\usepackage{amsmath,babel}
\usepackage[utf8]{inputenc}
\usepackage[includeheadfoot]{geometry}
\geometry{%
    headsep=.615cm,
    footskip=1cm,
    hmargin=2cm,
    vmargin=1.5cm
}
\usepackage{graphicx}
\usepackage[colorlinks=true,urlcolor=cyan]{hyperref}
\usepackage{blindtext}

\begin{document}
\pagenumbering{gobble}

\maketitle

\begin{figure}[h]
    \centering
    \includegraphics[scale=.5]{platzhalter}
\end{figure}

\section*{Aufgabenstellung}

Ein bereits vorhandener Roboter mit neigbarem Touchscreen soll um die Funktion erweitert werden, ein beliebiges auf dem Rahmen des Touchscreens aufliegendes Labyrinth (vorgegebener Rastergröße) mithilfe einer Metallkugel einlesen zu können. Der Roboter kann eine solche Kugel bereits von einem beliebigen Punkt auf der Oberfläche des Touchscreens aus auf eine ihm vorgegebene Position kippen; das Ziel ist es somit den Roboter mit der Kugel ein zuvor eingelesenes Labyrinth (auf optimalem Weg) lösen zu lassen.

\section*{Milestones}
\begin{enumerate}
    \item[\textbf{1.}] \textbf{Einarbeitung}
    \item[\textbf{2.}] \textbf{Einlesen des Labyrinths:} Algorithmus zur Lösungsfindung
        \begin{enumerate}
        \item[\textbf{a)}] formulieren.
        \item[\textbf{b)}] implementieren.
        \end{enumerate}
    \item[\textbf{3.}] \textbf{Fertigstellung}
\end{enumerate}

\end{document}