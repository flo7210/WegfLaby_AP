\documentclass{beamer}
\title{Wegfindung im Labyrinth (AP)}
\author{\texorpdfstring{Florian Nowak \& Yichuan Shen\\ Betreuer: Gero Plettenberg, Thomas Kloepfer}{Florian Nowak \& Yichuan Shen}}
\date{Wintersemester 2014/15}

%\usetheme{Goettingen}
\setbeamertemplate{navigation symbols}{}

\usepackage[utf8]{inputenc}

\usepackage[ngerman]{babel}
\usepackage{lmodern}
\usepackage{graphicx}
  \graphicspath{{bilder/}}
\usepackage{xparse}
  \NewDocumentCommand\blocking{m}{\textcolor{green}{\textit{#1}}}
  \NewDocumentCommand\blue{m}{{%
  	\usebeamercolor[fg]{frametitle}{#1}%
  	}}


\begin{document}
\maketitle

\begin{frame}[fragile,t]{Aufgabenstellung des Projekts}
Ein bereits vorhandener Roboter mit neigbarem Touchscreen soll etwas Neues lernen: Er soll ein beliebiges auf dem Rahmen seines Touchscreens aufliegendes Labyrinth (vorgegebener Rastergröße) mithilfe einer Metallkugel einlesen können. Der Roboter kann eine solche Kugel bereits auf eine ihm vorgegebene Position durch Kippen rollen lassen (und dort auch zum Stillstehen bringen). Das Ziel ist es somit, den Roboter ein zuvor eingelesenes Labyrinth mit der Kugel (auf optimalem Weg) lösen zu lassen.

\medskip\noindent
\blocking{Aufgabenstellung am Roboter erläutern!}
\end{frame}

\begin{frame}[fragile,t]{Vorstellung der Studierenden}
\blocking{Kurz vorstellen!}
\end{frame}

\begin{frame}[fragile,t]{Demonstration der Lösung des Labyrinths}
\blocking{\verb~string~-Darstellung des vorhandenen Labyrinths einlesen und Labyrinth lösen! Anschließend mit der Tiefensuche beginnen! (Letztere parallel zum weiteren Vortrag laufen lassen.)}
\end{frame}

\begin{frame}[fragile,t]{Praktikumsverlauf ausgehend von der ursprünglichen Planung}
\blue{Milestones:}

\smallskip
\begin{enumerate}
 \item \blue{Einarbeitung} \textit{(bis Anfang Dezember)}\blue{:} Vorgängercode und Funktionsfähigkeit des Roboters testen.
 \item \blue{Durchlaufen des Labyrinths} \textit{(bis Anfang Januar)}\blue{:}\\
 Durch Vorgabe eines Labyrinths als Datenstruktur soll die Kugel einen beliebig vorgegebenen Pfad durchlaufen können. Ferner wird ein optimaler Weg durch das Labyrinth berechnet.
 \item \blue{Erkennen des Labyrinths} \textit{(bis Ende Januar)}\blue{:}\\
 Durch Ablaufen eines realen Labyrinths soll die Struktur des Labyrinths erkannt und abgespeichert werden. Dafür wird ein Algorithmus entworfen und implementiert.
 \textcolor{gray}{
 \item[\textcolor{gray}{4.}] Fertigstellung \textit{(bis Anfang März)}:\\
Präsentation, Poster und Webseite fertig stellen; ggf. eine grafische Oberfläche erstellen.
}
\end{enumerate}
\end{frame}

\begin{frame}[fragile,t]{Praktikumsverlauf ausgehend von der ursprünglichen Planung}
Unseren Zeitplan konnten wir recht gut einhalten. Der dritte Milestone ist wie erwartet aufwendiger geworden, erreicht haben wir ihn erst nach den Klausuren (also Mitte Februar). Hinzugekommen sind (optische) Verbesserungen an der Hardware. \blocking{Letztere am Roboter zeigen!}

\smallskip
\begin{enumerate}
 \item[3.] \blue{Erkennen des Labyrinths (Tiefensuche)}
 \begin{enumerate}
 \item[(a)] Entwurf einer Routine zur Wanderkennung
 \item[(b)] Optimierung (hinsichtlich Geschwindigkeit)
 \end{enumerate}
 \item[\textcolor{gray}{4.}] \textcolor{gray}{Fertigstellung}
\end{enumerate}
\end{frame}

\begin{frame}[fragile,t]{Herangehensweise an die Problemstellung}
\begin{itemize}
\item Festlegung von \blue{Python (2.7)} als Programmiersprache
\item Auswahl des fortan zugrunde liegenden MCU-Codes, Einarbeitung und Reduktion auf die wesentlichen Routinen
\item Erweiterung des Roboter um eine serielle Schnittstelle, Herstellung der Kommunikation zwischen Computer und MCU
\begin{itemize}
\item Auslesen der Kugelposition
\item Schicken einer neuen Position
\item Schreiben einer Python-Klasse für den Touchscreen (\verb~Balancer~-Klasse)
\end{itemize}
\item Speicherung des Labyrinths als rechteckiger, einfacher Graph: Schreiben der \verb~Maze~-Klasse
\item \blocking{Bild zeichnen/tikz, einfügen!}
\end{itemize}
\end{frame}

\begin{frame}[fragile,t]{Ergebnis}
\begin{figure}
  \centering
  \includegraphics[scale=.0675]{roboter}
  %\caption{}
\end{figure}
\end{frame}

\begin{frame}[fragile,t]{Ergebnis}
\begin{enumerate}
\item Maze-Klasse kurz erklären (Labyrinth als Graph)
\item Balancer-Klasse/Kommunikation PC--TS (bzw. PC--MCU) erklären
\item Bestimmung des optimalen Wegs durch Breitensuche
\item Ablauf der Wanderkennung (Flowchart)
\item Optimierung
\begin{enumerate}
\item Die Funktion \verb~run(path, maze)~ berücksichtigt durch \verb~maze.get_skippables(path)~ erhaltene überspringbare Felder.
\item \verb~detect_walls~ lässt die Kugel auf dem letzten Nachbarn, falls dieser erreichbar war.
\item \verb~detect_maze~ berücksichtigt die Ausgabe von \verb~detect_walls~ und verhindert \glqq{}unnötige Pfade\grqq.
\end{enumerate}
\end{enumerate}
\end{frame}

\begin{frame}[fragile,t]{Aufgetretene Probleme}
Durch den unempfindlichen inneren Rand des Touchscreens dauert das Ausbalancieren in Randnähe durchschnittlich länger. Auch in verhältnismäig kleinen Labyrinthfeldern können Verzögerungen bei der Erkennung des Labyrinths auftreten. Die naheliegende Lösung beider Probleme ist eine Verbesserung der Hardware.
\end{frame}

\begin{frame}[fragile,t]{Ausblick/Reflektion}
Das Praktikum hat uns sehr gefallen und war ein guter Ausgleich zu unseren Vorlesungen.

\medskip\noindent
Wir denken nicht, dass sich der Roboter (bezogen auf unsere Aufgabenstellung) großartig verbessern lässt. Mögliche Erweiterungen:
\begin{itemize}
\item Labyrinthmaße experimentell bestimmen lassen
\item GUI schreiben
\textcolor{gray}{
\item[\textcolor{gray}{$\triangleright$}] Hardware erneuern und optimieren (beinhaltet Arbeit am MCU-Code)
}
\end{itemize}
\end{frame}


%\begin{frame}[fragile,t]{Thema}{Unterthema}
%\verb~print('Hello world!')~
%\end{frame}

\end{document}
