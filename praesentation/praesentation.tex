\documentclass{beamer}
\title{Wegfindung im Labyrinth (AP)}
\title{Wegfindung im Labyrinth (AP)}
\author{\texorpdfstring{Florian Nowak \& Yichuan Shen\\ Betreuer: Gero Plettenberg, Thomas Kloepfer}{Florian Nowak \& Yichuan Shen}}
\date{Wintersemester 2014/15}

\usetheme{Goettingen}
\setbeamertemplate{navigation symbols}{}

\usepackage[utf8]{inputenc}

\usepackage[ngerman]{babel}
\usepackage{lmodern}
\usepackage{graphicx}

\begin{document}
\maketitle

\begin{frame}[fragile,t]{}
Das Ergebnis des Praktikums ist in einer Präsentation vorzustellen. Diese besteht aus einem 20 minütigen Vortrag sowie einer 10 minütigen Demonstration der Lösung mit der Beantwortung eventueller Fragen. Die Zielgruppe der Präsentation ist ein fachlich vorgebildetes Publikum. Trotzdem sollte der Vortrag grundlegende Elemente der Aufgabenstellung erläutern.

\medskip\noindent
Der Vortrag beinhaltet folgende Punkte (siehe folgende Folien):
\end{frame}

\begin{frame}[fragile,t]{Aufgabenstellung des Projekts}
Text.
\end{frame}

\begin{frame}[fragile,t]{Vorstellung der Studierenden}
Text.
\end{frame}

\begin{frame}[fragile,t]{Herangehensweise an die Problemstellung}
Text.
\end{frame}

\begin{frame}[fragile,t]{Praktikumsverlauf ausgehend von der ursprünglichen Planung}
Text.
\end{frame}

\begin{frame}[fragile,t]{Ergebnis in sinnvoller Strukturierung}
Text.
\end{frame}

\begin{frame}[fragile,t]{Aufgetretene Probleme}
Ggf. mit Lösungsansatz.
\end{frame}

\begin{frame}[fragile,t]{Ausblick/Reflexion}
Mögliche Verbesserungen und Erweiterungen des Ergebnisses, welche ggf. in weiteren Praktika umgesetzt werden könnten.
\end{frame}

\begin{frame}[fragile,t]{}
Allgemein ist auf eine sinnvolle Strukturierung zu achten. Trennung in Hardware und Software ist zum Beispiel naheliegend. Oftmals dienen Programmablaufschemata und ausgewählte Codeausschnitte als gutes Mittel zur Veranschaulichung.
\end{frame}


%\begin{frame}[fragile,t]{Thema}{Unterthema}
%\verb~print('Hello world!')~
%\end{frame}

\end{document}