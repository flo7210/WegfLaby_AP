\documentclass[ngerman]{scrartcl}

\usepackage[landscape=true,margin=0cm]{geometry}

\usepackage[ngerman]{babel}
\usepackage[utf8]{inputenc}
\usepackage[default]{sourcesanspro}
\usepackage{tikz}
  \usetikzlibrary{external,shapes}
  \tikzexternalize

\begin{document}
\pagenumbering{gobble}

\tikzstyle{startend}=[
  circle,
  text centered,
  very thick,
  draw=red!50!black!70,
  fill=red!50!black!30,
  %top color=white,
  %bottom color=red!50!black!30
  ]
\tikzstyle{rounded}=[
  rounded rectangle,
  text centered,
  very thick,
  draw=lime!50!black!70,
  fill=lime!50!black!30,
  %top color=white,
  %bottom color=lime!50!black!30
  ]
\tikzstyle{kugel}=[
  rounded rectangle,
  text centered,
  text width=2.25cm,
  very thick,
  draw=blue!50!black!70,
  fill=blue!50!black!30,
  %top color=white,
  %bottom color=blue!50!black!30
  ]
\tikzstyle{block}=[
  rectangle,
  text centered,
  text width=2.25cm,
  very thick,
  draw=black!40,
  fill=black!24,
  %top color=white,
  %bottom color=black!30
  ]
\tikzstyle{decision}=[
  diamond,
  text centered,
  text width=2.25cm,
  very thick,
  draw=lime!50!black!70,
  fill=lime!50!black!30,
  %top color=white,
  %bottom color=lime!50!black!30
  ]
\tikzstyle{line}=[draw, thick, ->]

\begin{figure}[t]
  \centering
  \begin{tikzpicture}
  16.175
    \path[line] (16.825, -5.5) -- (16.825, -.525);
    \node at (-5.5, 5.5) [startend] (start) {Start};
    \node at (-5.5, 2.25) [block, text width=3cm] (generiere) {Generiere \verb~neighbors_stack~ von \verb~anchor~};
    \node at (-5.5, 0) [block, text width=3cm] (initialisierung) {Initialisierung: Balanciere Kugel auf \verb~anchor~ aus};
    \node at (0, 5.5) [block] (versucheerneut) {Versuche es erneut};
    \node at (5.5, 5.5) [decision] (nachbarzielist) {Ist Ziel ein Nachbar \textit{n}?};
    \node at (11, 5.5) [decision] (bereitszweimal) {Bereits zweimal versucht?};
    \node at (11, 2) [block, text width=2.5cm] (isteinewand) {Zwischen \verb~anchor~ und \textit{n} ist eine Wand};
    \node at (11, 0) [block, text width=2.5cm] (servoszurueck) {Setze die Servomotoren zurück};
	\node at (0, 0) [kugel] (kugelstart) {Kugel ist ausbalanciert};
	\node at (5.5, 0) [decision] (amziel) {Ist die Kugel am Ziel?};
	\node at (11, -4) [block] (gehezurueck) {Gehe zurück zu \verb~anchor~};
	\node at (16.5, 0) [kugel] (kugelende) {Kugel ist ausbalanciert};
	\node at (11, -5.5) [block] (isterreichbar) {\textit{n} ist erreichbar};
	\node at (5.5, -5.5) [decision] (istzielnachbar) {Ist Ziel ein Nachbar \textit{n}?};
	\node at (0, -5.5) [decision, text width=3cm] (stackleer) {Ist \verb~neighbors_stack~ leer?};
	\node at (16.5, -5.5) [block, text width=3.25cm] (weiterimtext) {Hole einen (neuen) Nachbarn \textit{n} aus \verb~neighbors_stack~ und schicke die Kugel dorthin};
	\node at (-5.5, -5.5) [startend] (ende) {Ende};
	\path[line] (start) -- (generiere);
	\path[line] (generiere) -- (initialisierung);
	\path[line] (initialisierung) -- (kugelstart);
	\path[line] (kugelstart) -- (amziel);
	\path[line] (amziel) -- node[left]{N} (nachbarzielist);
	\path[line] (nachbarzielist) -- node[above]{J} (bereitszweimal);
	\path[line] (bereitszweimal) -- node[left]{J} (isteinewand);
	\path[line] (isteinewand) -- (servoszurueck);
	\path[line] (servoszurueck) -- (kugelende);
	\path[line] (bereitszweimal) -- (11, 8.5) -- node[above]{N} (0, 8.5) -- (versucheerneut);
	\path[line] (versucheerneut) -- (kugelstart);
	\path[line] (nachbarzielist) -- node[above]{N} (versucheerneut);
	\path[line] (amziel) -- node[left]{J} (istzielnachbar);
	\path[line] (istzielnachbar) -- node[above]{N} (stackleer);
	\path[line] (stackleer) -- node[above]{J} (ende);
	\path[line] (stackleer) -- (0, -8.5) -- node[above]{N} (16.5, -8.5) -- (weiterimtext);
	\path[line] (istzielnachbar) -- node[above]{J} (isterreichbar);
	\path[line] (isterreichbar) -- (gehezurueck);
	\path[line] (gehezurueck) -- (11, -2.75) -- (16.175, -2.75) -- (16.175, -.525);
	\draw[thick, dashed, draw=blue!50!black!70] (kugelende) -- (16.5, -2.25) -- (0, -2.25) -- (kugelstart);
  \end{tikzpicture}
\end{figure}


\end{document}