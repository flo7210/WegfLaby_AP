\documentclass[ngerman]{scrartcl}
\title{Wegfindung im Labyrinth (AP)}
\author{
    Florian Nowak\footnote{Florian Nowak: B.\,Sc. Mathematik, 7. Fachsemester}\; und Yichuan Shen\footnote{Yichuan Shen: M.\,Sc. Mathematik, 1. Fachsemester}\\
    Betreuer: Gero Plettenberg, Thomas Kloepfer
}
\date{Wintersemester 2014/15}

\usepackage{amsmath,babel}
\usepackage[utf8]{inputenc}
\usepackage[includeheadfoot]{geometry}
\geometry{%
    headsep=.615cm,
    footskip=1cm,
    hmargin=2cm,
    vmargin=1.5cm
}
\usepackage{graphicx}
\usepackage[colorlinks=true,urlcolor=cyan]{hyperref}
\usepackage{blindtext}
\usepackage{paralist}

\begin{document}
\pagenumbering{gobble}

\maketitle

\begin{figure}[h]
    \centering
    \includegraphics[scale=.5]{platzhalter}
\end{figure}

\section*{Aufgabenstellung}

Ein bereits vorhandener Roboter mit neigbarem Touchscreen soll um die Funktion erweitert werden, ein beliebiges auf dem Rahmen des Touchscreens aufliegendes Labyrinth (vorgegebener Rastergröße) mithilfe einer Metallkugel einlesen zu können. Der Roboter kann eine solche Kugel bereits von einem beliebigen Punkt auf der Oberfläche des Touchscreens aus auf eine ihm vorgegebene Position kippen; das Ziel ist es somit den Roboter mit der Kugel ein zuvor eingelesenes Labyrinth (auf optimalem Weg) lösen zu lassen.

\section*{Milestones}
\begin{enumerate}
    \item \textbf{Einarbeitung:} Vorgängercode und Funktionsfähigkeit des Roboters testen.
    \item \textbf{Durchlaufen des Labyrinths:} Durch Vorgabe eines Labyrinths als Datenstruktur soll die Kugel einen beliebig vorgegebenen Pfad durchlaufen können. Ferner soll ein optimaler Weg durch das Labyrinth berechnet werden.
    \item \textbf{Erkennen des Labyrinths:} Durch Ablaufen eines realen Labyrinths soll die Struktur des Labyrinths erkannt und abgespeichert werden. Dafür soll ein Algorithmus entworfen und implementiert werden.
    \item \textbf{Fertigstellung:} Präsentation, Poster und Webseite fertig stellen. Gegebenfalls eine grafische Oberfläche erstellen.
\end{enumerate}

\end{document}