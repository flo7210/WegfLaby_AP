\documentclass[ngerman]{scrartcl}
\title{Wegfindung im Labyrinth (AP)}
\author{
    Florian Nowak\footnote{Florian Nowak: B.\,Sc. Mathematik, 7. Fachsemester}\; \& Yichuan Shen\footnote{Yichuan Shen: M.\,Sc. Mathematik, 1. Fachsemester}\\
    Betreuer: Gero Plettenberg, Thomas Kloepfer
}
\date{Wintersemester 2014/15}

\usepackage{amsmath,babel}
\usepackage[utf8]{inputenc}
\usepackage[includeheadfoot]{geometry}
\geometry{%
    headsep=0cm,
    footskip=1cm,
    hmargin=1cm,
    vmargin=0cm
}
\usepackage{graphicx}
\usepackage[colorlinks=true,urlcolor=cyan]{hyperref}

\begin{document}
\pagenumbering{gobble}

\maketitle

\begin{figure}[h]
    \centering
    \includegraphics[scale=.17]{photo.jpg}
\end{figure}

\section*{Aufgabenstellung}

Ein bereits vorhandener Roboter mit neigbarem Touchscreen soll etwas Neues lernen: Er soll ein beliebiges auf dem Rahmen seines Touchscreens aufliegendes Labyrinth (vorgegebener Rastergröße) mithilfe einer Metallkugel einlesen können. Der Roboter kann eine solche Kugel bereits auf eine ihm vorgegebene Position durch Kippen rollen lassen (und dort auch zum Stillstehen bringen). Das Ziel ist es somit, den Roboter ein zuvor eingelesenes Labyrinth mit der Kugel (auf optimalem Weg) lösen zu lassen.

\section*{Milestones}
\begin{enumerate}
    \item \textbf{Einarbeitung:} \textit{(bis Anfang Dezember)} Vorgängercode und Funktionsfähigkeit des Roboters testen.
    \item \textbf{Durchlaufen des Labyrinths:} \textit{(bis Anfang Januar)} Durch Vorgabe eines Labyrinths als Datenstruktur soll die Kugel einen beliebig vorgegebenen Pfad durchlaufen können. Ferner wird ein optimaler Weg durch das Labyrinth berechnet.
    \item \textbf{Erkennen des Labyrinths:} \textit{(bis Ende Januar)} Durch Ablaufen eines realen Labyrinths soll die Struktur des Labyrinths erkannt und abgespeichert werden. Dafür wird ein Algorithmus entworfen und implementiert.
    \item \textbf{Fertigstellung:} \textit{(bis Mitte März)} Präsentation, Poster und Webseite fertig stellen; ggf. eine grafische Oberfläche erstellen.
\end{enumerate}

\end{document}
